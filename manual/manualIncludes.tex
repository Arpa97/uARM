\usepackage[table]{xcolor}
\usepackage{longtable}
\usepackage{array}
\usepackage{multirow}
\usepackage{ulem}
\usepackage{bytefield}
\usepackage{framed}
\usepackage{adjustbox}
\usepackage{hhline}

% restore default italic emph behavior
\makeatletter
\DeclareRobustCommand{\myem}{%
	\@nomath\em
	\ifdim\fontdimen\@ne\font > \z@
	\eminnershape
	\else
	\itshape
\fi}
\DeclareTextFontCommand{\emph}{\myem}
\makeatother

\newcommand{\uarm}{$\mu$ARM}
\newcommand{\register}[1]{\textbf{#1}}
\newcommand{\field}[1]{\textit{#1}}
\newcommand{\addr}[1]{\texttt{#1}}

\newcommand{\centerinput}[1]{\vspace{5px}
\begin{adjustbox}{center}
	\input{#1}
\end{adjustbox}
\vspace{5px}}
\newcommand{\spinput}[1]{\centerinput{\specsd/tex/tables/#1}}

%bytefield macros
\newcommand{\colorbitbox}[3]{%
	\rlap{\bitbox{#2}{\color{#1}\rule{\width}{\height}}}%
	\bitbox{#2}{#3}}
% facilitates the creation of memory maps. Start address at the bottom,
% end address at the top.
% syntax:
% \memsection{end address}{start address}{height in lines}{bits of textbox}{text in box}
\newcommand{\memsection}[5]{%
% define the height of the memsection
	\bytefieldsetup{bitheight=#3\baselineskip}%
	\bitbox{#4}{#5}% print box with caption
	\bitbox[]{10}{%
		\addr{#1}% print end address
		\\
		% do some spacing
		\vspace{#3\baselineskip}
		\vspace{-2\baselineskip}
		\vspace{-#3pt}
		\addr{#2}% print start address
	}%
}
\newcommand{\vertbox}[1]{\rotatebox{90}{\footnotesize #1}}
