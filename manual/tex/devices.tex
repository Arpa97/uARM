\achapter{External Devices}\label{sec:man:device}

\asubsection{Installed Device Table}
Five words, from \texttt{0x20} to \texttt{0x30}, show the status of active devices. Each word represent a device line:
\\

\begin{tabular}{r|l}
\texttt{0x20} & Disks \\
\texttt{0x24} & Tapes \\
\texttt{0x28} & Network \\
\texttt{0x2C} & Printers \\
\texttt{0x30} & Terminals \\
\end{tabular}
\\

For each device line, if a specific device \emph{i} is enable, \emph{i}\textsuperscript{th} bit in representing word has value 1.

\asubsection{Device Registers}
Addresses \texttt{0x40} to \texttt{0x2C0} hold device registers, the behavior of this memory region is the same as $\mu$MPS machine's.

\asubsection{Pending Interrupt Bitmap}
Most of the interrupt lines are shared through all the devices of the same class, to identify which device is requesting for interrupt there are five registers from address \texttt{0x6FE0} to \texttt{0x6FF0} that hold a bitmap of interrupting devices per interrupt line.

This region is organized exactly as the Installed Device Table:
\\

\begin{tabular}{r|l}
\texttt{0x6FE0} & Disks \\
\texttt{0x6FE4} & Tapes \\
\texttt{0x6FE8} & Network \\
\texttt{0x6FEC} & Printers \\
\texttt{0x6FF0} & Terminals \\
\end{tabular}
\\

For each word, \emph{i} bit is set if \emph{i}\textsuperscript{th} device on that line is requesting for interrupt.

\asection{Interval Timer}\label{sec:man:intervalTimer}

Interval timer is decremented at each CPU cycle, when its value becomes 0 a software interrupt is thrown. It can be set to a desired value by writing its address in any privileged mode.


